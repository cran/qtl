\documentclass[10pt,letterpaper]{article}

\usepackage{times}     % times font
\usepackage{color}     % getting colored text
\usepackage{amsmath}
\usepackage{hyperref}

% revise margins
\setlength{\headheight}{0.0in}
\setlength{\topmargin}{-0.25in}
\setlength{\headsep}{0.0in}
\setlength{\textheight}{9.5in}
\setlength{\footskip}{0.35in}
\setlength{\oddsidemargin}{-0.25in}
\setlength{\evensidemargin}{-0.25in}
\setlength{\textwidth}{7.0in}
%\setlength{\parindent}{0pt}
%\setlength{\parsep}{12pt}

% font colors
\newcommand{\usercolor}{\color [named]{BlueViolet}}
\newcommand{\othercolor}{\color [named]{Mahogany}}
\newcommand{\lod}{\text{LOD}}

\begin{document}

\begin{center}
\rule{7.0in}{1mm} \vspace{0mm}

{\Large \textbf{A brief tour of R/qtl}} \vspace{4mm}

{\large Karl W Broman} \vspace{2mm}

Department of Biostatistics and Medical Informatics\\
University of Wisconsin -- Madison

\vspace{2mm}
\href{http://www.rqtl.org}{http://www.rqtl.org}
\vspace{2mm}

9 January 2009 % the date

\rule{7.0in}{1mm} 
\end{center}

\noindent \textbf{Overview of R/qtl} \vspace{6pt}

R/qtl is an extensible, interactive environment for mapping
quantitative trait loci (QTL) in experimental crosses. It is
implemented as an add-on package for the freely available and widely
used statistical language/software R (see
\href{http://www.r-project.org}{www.r-project.org}). The development
of this software as an add-on to R allows us to take advantage of the
basic mathematical and statistical functions, and powerful graphics
capabilities, that are provided with R. Further, the user will benefit
by the seamless integration of the QTL mapping software into a general
statistical analysis program.  Our goal is to make complex QTL mapping
methods widely accessible and allow users to focus on modeling rather
than computing.

A key component of computational methods for QTL mapping is the hidden
Markov model (HMM) technology for dealing with missing genotype
data. We have implemented the main HMM algorithms, with allowance for
the presence of genotyping errors, for backcrosses, intercrosses, and
phase-known four-way crosses.

The current version of R/qtl includes facilities for estimating
genetic maps, identifying genotyping errors, and performing single-QTL
genome scans and two-QTL, two-dimensional genome scans, by interval
mapping (with the EM algorithm), Haley-Knott regression, and multiple
imputation. All of this may be done in the presence of covariates
(such as sex, age or treatment). One may also fit higher-order QTL models
by multiple imputation and Haley-Knott regression.

R/qtl is distributed as source code for Unix or compiled code for
Windows or Mac OS X.  R/qtl is released under the GNU General Public
License. To download the software, you must agree to the terms in that
license.


\vspace{12pt}
\noindent \textbf{Overview of R} \vspace{6pt}

R is an open-source implementation of the S language. As described on
the R-project homepage
(\href{http://www.r-project.org}{www.r-project.org}):

\begin{quote}
R is a system for statistical computation and graphics.  It consists
of a language plus a run-time environment with graphics, a debugger,
access to certain system functions, and the ability to run programs
stored in script files.

The core of R is an interpreted computer language which allows
branching and looping as well as modular programming using
functions. Most of the user-visible functions in R are written in
R. It is possible for the user to interface to procedures written in
the C, C++, or FORTRAN languages for efficiency. The R distribution
contains functionality for a large number of statistical
procedures. Among these are: linear and generalized linear models,
nonlinear regression models, time series analysis, classical
parametric and nonparametric tests, clustering and smoothing. There is
also a large set of functions which provide a flexible graphical
environment for creating various kinds of data
presentations. Additional modules are available for a variety of
specific purposes.
\end{quote} 

R is freely available for Windows, Unix and Mac OS X, and may be
downloaded from the Comprehensive R Archive Network (CRAN;
cran.r-project.org).

Learning R may require a formidable investment of time, but it will
definitely be worth the effort. Numerous free documents on getting
started with R are available on CRAN. In additional, several books are
available.  The most important book on R is Venables and Ripley (2002)
\emph{Modern Applied Statistics with S}, 4th
edition. Dalgaard (2002) \emph{Introductory Statistics with
  R} provides a more gentle introduction.


\vspace{12pt}
\noindent \textbf{Citation for R/qtl} \vspace{6pt}

To cite R/qtl in publications, use

\begin{quote}
Broman KW, Wu H, Sen S, Churchill GA (2003) R/qtl: QTL mapping in
experimental crosses.  Bioinformatics 19:889-890
\end{quote}

\newpage

\noindent \textbf{Selected R/qtl functions} 

%\renewcommand{\arraystretch}{0.9}
\noindent \begin{tabular}{lll} 
\hspace*{35mm} & \hspace*{25mm} & \hspace*{103mm} \\ \hline
\textbf{Sample data} 
& badorder & An intercross with misplaced markers \\
& bristle3 & Data on bristle number for Drosophila chromosome 3 \\
& bristleX & Data on bristle number for Drosophila X chromosome \\
& fake.4way & Simulated data for a 4-way cross \\
& fake.bc & Simulated data for a backcross \\
& fake.f2 & Simulated data for an F$_2$ intercross \\
& hyper & Backcross data on salt-induced hypertension \\ 
& listeria & Intercross data on Listeria monocytogenes
susceptibility \\ 
& map10 & A genetic map modeled after the mouse genome (10 cM
spacing)\\ 
\hline

\textbf{Input/output} 
& read.cross & Read data for a QTL
experiment \\
& write.cross & Write data for a QTL experiment to a file \\
\hline 

\textbf{Simulation} 
& sim.cross & Simulate a QTL experiment \\
& sim.map & Generate a genetic map \\ 
\hline

\textbf{Summaries} 
& qtlversion & Gives the version number of installed R/qtl package \\
& plot.cross & Plot various features of a cross object \\
& plot.missing & Plot grid of missing genotypes \\
& geno.image   & Plot grid with colored pixels representing different
genotypes \\
& plot.pheno   & Histogram or bar plot of a phenotype \\
& plot.info & Plot the proportion of missing genotype data \\
& summary.cross & Print summary of QTL experiment \\ 
& summary.map & Print summary of a genetic map \\ 
& \multicolumn{2}{l}{nchr, nind, nmar, nphe, totmar, nmissing, ntyped} \\ 
& find.pheno & Find the column number for a particular phenotype \\
& find.marker & Find the marker closest to a specified position \\
& find.flanking & Find the markers flanking a particular position \\
& find.pseudomarker & Find the name of the grid position closest to a
particular position \\
& find.markerpos & Find the map positions of a marker \\
\hline

\textbf{Data manipulation} 
& clean.cross & Remove intermediate calculations from a cross \\
& drop.markers & Remove a list of markers \\
& drop.nullmarkers & Remove markers without data \\
& fill.geno & Fill in holes in genotype data by 
imputation or Viterbi \\ 
& strip.partials & Replace partially informative genotypes with
missing values \\
& pull.map & Pull out the genetic map from a cross \\
& pull.geno & Pull out the genotype data as a matrix \\
& pull.pheno & Pull out a phenotype \\
& replace.map & Replace the genetic map of a cross \\
& jittermap & Jitter marker positions slightly so that no two coincide \\
& subset.cross & Select a subset of chromosomes and/or individuals from
a cross \\
& c.cross & Combine two crosses into one object \\
& switch.order & Switch the order of markers on a chromosome \\ 
& movemarker & Move a marker from one chromosome to another \\ 
\hline 

\textbf{HMM engine} 
& argmax.geno & Reconstruct underlying genotypes by the Viterbi
algorithm \\ 
& calc.genoprob & Calculate conditional genotype probabilities \\ 
& sim.geno & Simulate genotypes given observed marker data \\ 
\hline

\textbf{Diagnostics} 
& geno.table & Create table of genotype distributions \\ 
& geno.crosstab & Create cross-tabulation of genotypes at two markers \\
& checkAlleles & Identify markers with potentially switched alleles \\
& calc.errorlod & Calculate Lincoln \& Lander (1992) error LOD scores \\
& top.errorlod & List genotypes with highest error LOD values \\ 
& plot.geno & Plot observed genotypes, flagging likely errors \\ 
& comparecrosses & Compare two cross objects, to see if they are the same \\
& comparegeno & Calculate proportion of matching genotypes for each
pair of individuals \\
\hline

\end{tabular}

\newpage

\noindent \textbf{Selected R/qtl functions (continued)} 

%\renewcommand{\arraystretch}{0.9}
\noindent \begin{tabular}{lll} 
\hspace*{35mm} & \hspace*{25mm} & \hspace*{103mm} \\ 
\hline

\textbf{Genetic mapping} 
& est.rf & Estimate pairwise recombination fractions \\ 
& plot.rf & Plot recombination fractions \\ 
& est.map & Estimate genetic map \\
& plot.map & Plot genetic map(s) \\
& summary.map & Print summary of a genetic map \\
& ripple & Assess marker order by permuting groups of adjacent
markers \\
& summary.ripple & Print summary of ripple output \\ 
& compareorder & Compare two orderings of markers on a chromosome \\
& tryallpositions & Test all possible positions for a marker \\
\hline

\textbf{QTL mapping} 
& scanone & Genome scan with a single QTL model \\
& scantwo & Two-dimensional genome scan with a two-QTL model \\
& lodint & Calculate a LOD support interval \\
& bayesint & Calculate an approximate Bayes credible interval \\
& scanoneboot & Non-parametric bootstrap to obtain a confidence
interval for QTL location \\
& plot.scanone & Plot output for a one-dimensional genome scan \\
& add.threshold & Add a horizontal line at a LOD threshold to a genome scan plot \\
& plot.scantwo & Plot output for a two-dimensional genome scan \\ 
& summary.scanone & Print summary of scanone output \\
& summary.scantwo & Print summary of scantwo output \\ 
& max.scanone & Maximum peak in scanone output \\
& max.scantwo & Maximum peak in scantwo output \\
& effectplot & Plot phenotype means of genotype groups defined
by 1 or 2 markers \\
& effectscan & Plot estimated QTL effects across the whole genome \\
& plot.pxg & Like effectplot, but as a dot plot of the phenotypes \\ 
\hline

\textbf{Multiple QTL models} 
& makeqtl & Make a qtl object for use by fitqtl \\ 
& fitqtl & Fit a multiple QTL model \\
& summary.fitqtl & Get summary of the result of fitqtl \\
& scanqtl & Perform a multi-dimensional genome scan \\
& refineqtl & Refine the QTL locations in a multiple QTL model \\
& plotLodProfile & Plot 1-dimensional LOD profiles for a multiple QTL
model \\
& addqtl & Scan for an additional QTL, in a 
multiple-QTL model \\
& addpair & Scan for an additional pair of QTL, in a 
multiple-QTL model \\
& addint & Add pairwise interactions, 
one at a time, in a multiple-QTL model \\
& summary.qtl & Print a summary of a QTL object \\
& plot.qtl & Plot the QTL locations on the genetic map \\
& addtoqtl & Add to a QTL object \\
& dropfromqtl & Drop a QTL from a QTL object \\
& replaceqtl & Replace a QTL location in a QTL object with a different
position \\
& reorderqtl & Reorder the QTL in a QTL object \\
& cim & A (relatively crude) implementation of Composite Interval
Mapping \\
& stepwiseqtl & Stepwise selection for multiple QTL \\
& calc.penalties & Calculate penalties for use with stepwiseqtl \\
& plotModel & Plot a graphical representation of a multiple-QTL model \\
\hline
\end{tabular} 
\newpage






\noindent \textbf{Preliminaries} \vspace{6pt}

\noindent Use of the R/qtl package requires considerable knowledge of
the R language/environment.  We hope that the examples presented here
will be understandable with little prior knowledge of R, especially
because we neglect to explain the syntax of R.  Several books, as well
as some free documents, are available to assist the user in learning
R; see the R project website cited above.  We assume here that the
user is running either Windows or Mac OS X.

\begin{enumerate}

\item To start R, double-click its icon.  

\item To exit, type:

\usercolor \verb-q()- \normalcolor

Click yes or no to save or discard your work.

\item R keeps all of your work in RAM.  If R should crash, all will be
  lost, and you will have to start from the beginning.  The function
  \verb-save.image- can be used to save your work to disk as you go
  along, so that, should R crash, you won't have to start from
  scratch.  You would type:

\usercolor \verb|save.image()| \normalcolor 

\item Load the R/qtl package:

\usercolor \verb|library(qtl)| \normalcolor

\item View the objects in your workspace:

\usercolor \verb|ls()| \normalcolor

\item The best way to get help on the functions and data sets in R
(and in R/qtl) is via the html version of the help files. One way to
get access to this is to type 

\usercolor \verb-help.start()- \normalcolor

This should open a browser with the main help menu.  
If you then click on \othercolor Packages \normalcolor $\rightarrow$
\othercolor qtl\normalcolor , you can see all of the available
functions and datasets in R/qtl.  For example, look at the help file
for the function \verb-read.cross-.

An alternative method to view this help file is to type one of the
following: 

\usercolor \verb|help(read.cross)| \\
\verb|?read.cross| \normalcolor

The html version of the help files are somewhat easier to read, and
allow use of hotlinks between different functions.  

%You can create a file \othercolor \verb-"c:\.Rprofile"- \normalcolor
%(\othercolor \verb-~/.Rprofile- \normalcolor in Mac OS X) containing any
%R code to be executed whenever R is started.  The command \usercolor
%\verb-library(qtl)- \normalcolor 
%is a good candidate for
%placement in such a file.

\item All of the code in this tutorial is available as a file from
  which you may copy and paste into R, if you prefer that to typing.
  Type the following within R to get access to the file:

\usercolor \verb-url.show("http://www.rqtl.org/rqtltour.R")-
\normalcolor

\end{enumerate}


%\vspace{12pt}
\noindent \textbf{Data import} \vspace{6pt}

\noindent A difficult first step in the use of most data analysis
software is the import of data.  With R/qtl, one may import data in
several different formats by use of the function \verb-read.cross-.
(Example data files are available at
\href{http://www.rqtl.org/sampledata}{www.rqtl.org/sampledata}.)  The
internal data structure used by R/qtl is rather complicated, and is
described in the help file for \verb-read.cross-.  (Also see Example
6 on page~\pageref{example6}.) We won't discuss data import any further here, except to
say that the comma-delimited format (\verb-"csv"-) is recommended.  If
you have trouble importing data, send an email to Karl Broman
(\verb-kbroman@biostat.wisc.edu-), attaching examples of your data files.
(Such data will be kept confidential.)



\vspace{12pt}
\noindent \textbf{Example 1: Hypertension} \vspace{6pt}
\nopagebreak

\noindent As a first example, we consider data from an experiment on
hypertension in the mouse (Sugiyama et al., Genomics 71:70-77, 2001),
kindly provided by Bev Paigen and Gary Churchill.

\begin{enumerate}

\item First, get access to the data, see that it is in your
workspace, and view its help file.  These data are included with the
R/qtl package, and so you can get access to the data with the function
\verb-data()-.  (Remember that you first need to load the R/qtl
package via \verb-library(qtl)-.)

\usercolor \verb|data(hyper)| \\
\verb|ls()| \\
\verb|?hyper| \normalcolor

\item We will postpone discussion of the internal data structure used
by R/qtl until later.  For now we'll just say that the data
\verb-hyper- has ``class'' \verb-"cross"-.  The function
\verb-summary.cross- prints summary information on such data.  We can
call that function directly, or we may simply use \verb-summary- and
the data is sent to the appropriate function according to its class.

\usercolor \verb|summary(hyper)| 
\normalcolor

Several other utility functions are available for getting summary
information on the data.  Hopefully these are self-explanatory.  

\usercolor
\verb|nind(hyper)| \\
\verb|nphe(hyper)| \\
\verb|nchr(hyper)| \\
\verb|totmar(hyper)| \\
\verb|nmar(hyper)| \normalcolor

\item Plot a summary of these data.

\usercolor \verb|plot(hyper)| \normalcolor

In the upper left, black pixels indicate missing genotype data.  Note
that one marker has no genotype data.  In the upper right, the genetic
map of the markers is shown.  In the lower left, a histogram of the
phenotype is shown.

The Windows version of R has a slick method for recording
graphs, so that one may page up and down through a series of plots.
To initiate this, click (on the menu bar) \othercolor History
\normalcolor $\rightarrow$ \othercolor Recording\normalcolor .

We may plot the individual components of the above multi-plot figure
as follows.

\usercolor 
\verb|plot.missing(hyper)| \\
\verb|plot.map(hyper)| \\
\verb|plot.pheno(hyper, pheno.col=1)| %$
\normalcolor

We can plot the genetic map with marker names, but they can be rather
difficult to read.  The following code plots the map with marker names
for chr 1, 4, 6, 7 and 15.

\usercolor
\verb|plot.map(hyper, chr=c(1, 4, 6, 7, 15), show.marker.names=TRUE)|
\normalcolor

\item Note the odd pattern of missing data; we may make this missing
data plot with the individuals ordered according to the value of their
phenotype.

\usercolor 
\verb|plot.missing(hyper, reorder=TRUE)| 
\normalcolor

We see that, for most markers, only individuals with extreme
phenotypes were genotyped.  At many markers (in regions of interest),
markers were typed only on recombinant individuals.

\item The function \verb-drop.nullmarkers- may be used to remove
markers that have no genotype data (such as the marker on chr
14).  A call to \verb-totmar- will show that there are now 173 markers
(rather than 174, as there were initially).  

\usercolor
\verb|hyper <- drop.nullmarkers(hyper)| \\
\verb|totmar(hyper)| \normalcolor

\item Estimate recombination fractions between all pairs of markers,
and plot them. This also calculates LOD scores for the test of
H$_0{:} \; r=1/2$.  The plot of the recombination fractions can be
either with recombination fractions in the upper part and LOD scores
below, or with just recombination fractions or just LOD scores.  Note
that red corresponds to a small recombination fraction or a big LOD
score, while blue is the reverse.  Gray indicates missing values.

\usercolor \verb|hyper <- est.rf(hyper)| \\
\verb|plot.rf(hyper)| \\
\verb|plot.rf(hyper, chr=c(1,4))| \normalcolor

There are some very strange patterns in the recombination fractions,
but this is due to the fact that some markers were typed largely on
recombinant individuals.

For example, on chr 6, the tenth marker shows a high recombination
fraction with all other markers on the chromosome, but a plot of the
missing data shows that this marker was typed only on a selected
number of individuals (largely those showing recombination events
across the interval).

\usercolor \verb|plot.rf(hyper, chr=6)| \\
\verb|plot.missing(hyper, chr=6)| \normalcolor

\item Re-estimate the genetic map (keeping the order of markers
fixed), and plot the original map against the newly estimated one. 

\usercolor \verb|newmap <- est.map(hyper, error.prob=0.01)| \\
\verb|plot.map(hyper, newmap)| \normalcolor

We see some map expansion, especially on chr 6, 13 and 18.  It
is questionable whether we should replace the map or not.  Keep in
mind that the previous map locations are based on a limited number of
meioses.  If one wished to replace the genetic map with the estimated
one, it could be done as follows:

\usercolor
\verb|hyper <- replace.map(hyper, newmap)| \normalcolor

This replaces the map in the \verb-hyper- data with \verb-newmap-.


\item We now turn to the identification of genotyping errors.  In the
following, we calculate the error LOD scores of Lincoln and Lander
(1992).  A LOD score is calculated for each individual at each marker;
large scores indicate likely genotyping errors.

\usercolor 
\verb|hyper <- calc.errorlod(hyper, error.prob=0.01)| 
\normalcolor

This calculates the genotype error LOD scores and inserts them into
the \verb-hyper- object.

The function \verb-top.errorlod- gives a list of genotypes that may be
in error.  Error LOD scores $<$ 4 can probably be ignored.

\usercolor 
\verb|top.errorlod(hyper)| 
\normalcolor

Note that the results will be different, depending on whether you used
\verb-replace.map- above.  If you did, you will get an indication of
potential errors on chr 16 (and a few on chr 13).  If you didn't, you will get a very long
list of potential errors on chr 1, 11, 15, 16 and 17.

\item The function \verb-plot.geno- may be used to inspect the
observed genotypes for a chromosome, with likely genotyping errors
flagged.  Of course, it's difficult to look at too many individuals at
once.  Note that white = AA and black = AB (for a backcross).  

\usercolor
\verb|plot.geno(hyper, chr=16, ind=c(24:34, 71:81))|
\normalcolor

We don't have any utilities for fixing any apparent errors; it would
be best to go back to the raw data.  (Of course, you should edit a
copy of the file; never discard the primary data.)

\item The function \verb-plot.info- plots a measure of the proportion
of missing genotype information in the genotype data.  The missing
information is calculated in two ways: as entropy, or via the
variance of the conditional genotypes, given the observed marker
data.  (See the help file, using \verb-?plot.info-.)  

\usercolor
\verb|plot.info(hyper)| \\
\verb|plot.info(hyper, chr=c(1,4,15))| \\
\verb|plot.info(hyper, chr=c(1,4,15), method="entropy")| \\
\verb|plot.info(hyper, chr=c(1,4,15), method="variance")| 
\normalcolor


\item We now, finally, get to QTL mapping.  

The core of R/qtl is a set of functions which make use of the hidden
Markov model (HMM) technology to calculate QTL genotype probabilities,
to simulate from the joint genotype distribution and to calculate the
most likely sequence of underlying genotypes (all conditional on the
observed marker data).  This is done in a quite general way, with
possible allowance for the presence of genotyping errors.  Of course, 
for convenience we assume no crossover interference.  

The function \verb-calc.genoprob- calculates QTL genotype
probabilities, conditional on the available marker data. These are
needed for most of the QTL mapping functions.  The argument
\verb-step- indicates the step size (in cM) at which the probabilities
are calculated, and determines the step size at which later LOD scores
are calculated.

\usercolor
\verb|hyper <- calc.genoprob(hyper, step=1, error.prob=0.01)| \normalcolor

We may now use the function \verb-scanone- to perform a single-QTL
genome scan with a normal model.  We may use maximum likelihood via
the EM algorithm (Lander and Botstein 1989) or use Haley-Knott
regression (Haley and Knott 1992).

\usercolor
\verb|out.em <- scanone(hyper)| \\
\verb|out.hk <- scanone(hyper, method="hk")|
\normalcolor  

We may also use the multiple imputation method of Sen and Churchill
(2001).  This requires that we first use \verb-sim.geno- to simulate
from the joint genotype distribution, given the observed marker data.  
Again, the argument \verb-step- indicates the step size at which the
imputations are performed and determines the step size at which LOD
scores will be calculated.  The \verb-n.draws- indicates the number of
imputations to perform.  Larger values give more precise results but
require considerably more computer memory and computation time.

\usercolor
\verb|hyper <- sim.geno(hyper, step=2, n.draws=16, error.prob=0.01)| \\
\verb|out.imp <- scanone(hyper, method="imp")|
\normalcolor

\item The output of scanone has class
  \verb-"scanone"-; the function \verb-summary.scanone- displays the
  maximum LOD score on each chromosome for which the LOD exceeds a
  specified threshold.

\usercolor
\verb|summary(out.em)| \\
\verb|summary(out.em, threshold=3)| \\
\verb|summary(out.hk, threshold=3)| \\
\verb|summary(out.imp, threshold=3)| 
\normalcolor

\item The function \verb-max.scanone- returns just the highest peak
from output of \verb-scanone-.

\usercolor
\verb|max(out.em)| \\
\verb|max(out.hk)| \\
\verb|max(out.imp)| 
\normalcolor

\item We may also plot the results.  \verb-plot.scanone- can plot up
to three genome scans at once, provided that they conform
appropriately.  Alternatively, one may use the argument \verb-add-.

\usercolor
\verb|plot(out.em, chr=c(1,4,15))| \\
\verb|plot(out.em, out.hk, out.imp, chr=c(1,4,15))|  \\
\verb|plot(out.em, chr=c(1,4,15))| \\
\verb|plot(out.hk, chr=c(1,4,15), col="blue", add=TRUE)| \\
\verb|plot(out.imp, chr=c(1,4,15), col="red", add=TRUE)|
\normalcolor

\item The function \verb-scanone- may also be used to perform a
permutation test to get a genome-wide LOD significance threshold.
For Haley-Knott regression, this can be quite fast.

\usercolor 
\verb|operm.hk <- scanone(hyper, method="hk", n.perm=1000)|
\normalcolor

The permutation output has class \verb-"scanoneperm"-.  The function
\verb-summary.scanoneperm- can be used to get significance
thresholds.  

\usercolor 
\verb|summary(operm.hk, alpha=0.05)| 
\normalcolor

In addition, if the permutations results are included in a call to
\verb-summary.scanone-, you can estimated genome-scan-adjusted
p-values for inferred QTL, and can get a report of all chromosomes
meeting a certain significance level, with the corresponding LOD
threshold calculated automatically.

\usercolor
\verb|summary(out.hk, perms=operm.hk, alpha=0.05, pvalues=TRUE)|
\normalcolor

%\item Recall that the pattern of missing genotypes in the
%  \verb-hyper- data.  At most markers, only the phenotypi extremes
%  were genotyped.  We may use the function \verb-nmissing- to obtain
%  the number of missing genotypes for each individual, and use
%  \verb-hist- to plot a histogram.
%
%\usercolor
%\verb|nm <- nmissing(hyper)| \\
%\verb|hist(nm, breaks=50)| \\
%\verb|rug(nm)|
%\normalcolor
%
%The argument \verb-breaks- is used to specify the number of bins in
%the histogram.  The funtion \verb-rug- is used to add tick marks at
%the bottom at the data points.  
%
%In this figure, we see that individuals were missing genotypes at
%either $<$ 30 or $>$ 120.  
%
%This calls into question the permutation test we ran above, as the
%phenotypes were permuted across all individuals.  It would be better
%to perform a stratified permutation test, permuting individuals within
%the more completely genotyped group and separately permuting those
%within the less completely genotyped group.
%
%This may be done with the \verb-perm.strata- argument to scanone,
%which will define strata in which permutations will be performed.  The
%stratified permutation test may be performed as follows.
%
%\usercolor 
%\verb|operm2.hk <- scanone(hyper, method="hk", n.perm=1000,| \\
%\verb|                     perm.strata=(nm > 100))|
%\normalcolor
%
%We may use \verb-summary- again to get a LOD threshold.  This is much
%larger than that calculated from the unstratified permutation test.
%
%\usercolor
%\verb|summary(operm2.hk, alpha=0.05)|
%\normalcolor



\item We should mention at this point that the function
\verb-save.image- may be used to save your workspace to disk.  If R
crashes, you will wish you had used this.  

\usercolor \verb|save.image()| \normalcolor

\item The function \verb-scantwo- performs a two-dimensional genome
scan with a two-QTL model.  For every pair of positions, it calculates
a LOD score for the full model (two QTL plus interaction) and a LOD
score for the additive model (two QTL but no interaction).  This be
quite time consuming, and so you may wish to do the calculations on a
coarser grid.

\usercolor
\verb|hyper <- calc.genoprob(hyper, step=5, error.prob=0.01)| 
\\
\verb|out2.hk <- scantwo(hyper, method="hk")| 
\normalcolor

One can also use \verb-method="em"- or \verb-method="imp"-, but they
are even more time consuming.

\item \label{scantwo} The output of \verb-scantwo- has class \verb-"scantwo"-; there
are functions for obtaining summaries and plots, of course.

The summary function considers each pair of chromosomes, and
calculates the maximum LOD score for the full model ($M_f$) and the maximum
LOD score for the additive model ($M_a$).  These two models are allowed
to be maximized at different positions.   We futher calculate a LOD
score for a test of epistasis, $M_i = M_f - M_a$, and two LOD scores
that concern evidence for a second QTL: $M_{fv1}$ is the LOD score
comparing the full model to the best single-QTL model and $M_{av1}$ is
the LOD score comparing the additive model to the best single-QTL
model.  

In the summary, we must provide five thresholds, for $M_f$, 
$M_{fv1}$, $M_i$, $M_a$, and $M_{av1}$, respectively.  Call these $T_f$,
$T_{fv1}$, $T_i$, $T_a$, and $T_{av1}$.  We then report those pairs of
chromosomes for which at least one of the following holds:
\begin{itemize}
\item $M_f \ge T_f$ and ($M_{fv1} \ge T_{fv1}$ or $M_i \ge T_i$)
\item $M_a \ge T_a$ and $M_{av1} \ge T_{av1}$
\end{itemize}

The thresholds can be obtained by a permutation test (see below), but
this is extremely time-consuming.  For a mouse backcross, we suggest
the thresholds (6.0, 4.7, 4.4, 4.7, 2.6) for the full,
conditional-interactive, interaction, additive, and
conditional-additive LOD scores, respectively.  For a mouse
intercross, we suggest the thresholds (9.1, 7.1, 6.3, 6.3, 3.3) for the
full, conditional-interactive, interaction, additive, and
conditional-additive LOD scores, respectively.  These were obtained by
10,000 simulations of crosses with 250 individuals, markers at a 10 cM
spacing, and analysis by Haley-Knott regression.

\usercolor
\verb|summary(out2.hk, thresholds=c(6.0, 4.7, 4.4, 4.7, 2.6))|
\normalcolor

The appropriate decision rule is not yet completely clear.  I am
inclined to ignore $M_i$ and to choose genome-wide thresholds for the
other four based on a permutation, using a common significance level
for all four.  $M_i$ would be ignored if we gave it a very large
threshold, as follows.

\usercolor
\verb|summary(out2.hk, thresholds=c(6.0, 4.7, Inf, 4.7, 2.6))|
\normalcolor

\item Plots of \verb-scantwo- results are created via
  \verb-plot.scantwo-.  

\usercolor
\verb|plot(out2.hk)| \\
\verb|plot(out2.hk, chr=c(1,4,6,15))|
\normalcolor

By default, the upper-left triangle contains epistasis LOD scores and
the lower-right triangle contains the LOD scores for the full model.
The color scale on the right indicates separate scales for the
epistasis and joint LOD scores (on the left and right, respectively).

\item The function \verb-max.scantwo- returns the two-locus positions
with the maximum LOD score for the full and additive models.

\usercolor
\verb|max(out2.hk)|
\normalcolor

\item One may also use \verb-scantwo- to perform permutation tests in
order to obtain genome-wide LOD significance thresholds.  These can be
extremely time consuming, though with the Haley-Knott regression and
multiple imputation methods, there is a trick that may be used in some
cases to dramatically speed things up.  So we'll try 100 permutations
by the Haley-Knott regression method and hope that your computer is
sufficiently fast.

\usercolor
\verb|operm2.hk <- scantwo(hyper, method="hk", n.perm=100)|
\normalcolor

We can again use \verb-summary- to get LOD thresholds.

\usercolor
\verb|summary(operm2.hk)|
\normalcolor

And again these may be used in the summary of the \verb-scantwo-
output to calculate thresholds and p-values.  If you want to ignore
the LOD score for the interaction in the rule about what chromosome
pairs to report, give $\alpha=0$, corresponding to a threshold
$T=\infty$. 

\usercolor
\verb|summary(out2.hk, perms=operm2.hk, pvalues=TRUE,| \\
\verb|        alphas=c(0.05, 0.05, 0, 0.05, 0.05))|
\normalcolor

You can't really trust these results.  Haley-Knott regression performs
poorly in the case of selective genotyping (as with the \verb-hyper-
data).  Standard interval mapping or imputation would be better, but
Haley-Knott regression has the advantage of speed, which is the reason
we use it here.

\item Finally, we consider the fit of multiple-QTL models.  Currently,
only multiple imputation and Haley-Knott regression has been
implemented.  We use multiple imputation here, as Haley-Knott
regression performs poorly in the case of selective genotyping,
which was used for the \verb-hyper- data.  We first create a QTL
object using the function \verb-makeqtl-, with five QTL at
specified, fixed positions.

\usercolor
\verb|chr <- c(1, 1, 4, 6, 15)| \\
\verb|pos <- c(50, 76, 30, 70, 20)| \\
\verb|qtl <- makeqtl(hyper, chr, pos)|
\normalcolor

Finally, we use the function \verb-fitqtl- to fit a model with five
QTL, and allowing the QTL on chr 6 and 15 to interact.

\usercolor
\verb|my.formula <- y ~ Q1 + Q2 + Q3 + Q4 + Q5 + Q4:Q5| \\ 
\verb|out.fitqtl <- fitqtl(hyper, qtl=qtl, formula=my.formula)| \\ %$
\verb|summary(out.fitqtl)|
\normalcolor

See Example 5 (page~\pageref{example5}) for a thorough discussion of
the multiple QTL mapping methods in R/qtl.

\item You may wish to clean up your workspace before we move on to the
next example.

\usercolor 
\verb|ls()| \\ 
\verb|rm(list=ls())| 
\normalcolor

\end{enumerate}







\vspace{12pt}
\noindent \textbf{Example 2: Genetic mapping} \vspace{6pt}
\nopagebreak

\noindent R/qtl includes some utilities for estimating genetics maps
and checking marker orders.  In this example, we describe the use of
these utilities.

\begin{enumerate}

\item Get access to some sample data.  This is simulated data with
some errors in marker order.

\usercolor 
\verb|data(badorder)|  \\
\verb|summary(badorder)| \\
\verb|plot(badorder)|
\normalcolor

\item Estimate recombination fractions between all pairs of markers,
and plot them. 

\usercolor \verb|badorder <- est.rf(badorder)| \\
\verb|plot.rf(badorder)| \normalcolor

It appears that markers on chr 2 and 3 have been switched.

Also note that, if we look more closely at the recombination fractions
for chr 1, there seem to be some errors in marker order.

\usercolor \verb|plot.rf(badorder, chr=1)| \normalcolor

\item Re-estimate the genetic map.  

\usercolor
\verb|newmap <- est.map(badorder, verbose=TRUE)| \\
\verb|plot.map(badorder, newmap)|
\normalcolor

This really shows the problems on chr 2 and 3.

\item Fix the problems on chr 2 and 3.
First, we look more closely at the recombination fractions for these
chromosoems

\usercolor \verb|plot.rf(badorder, chr=2:3)| \normalcolor

We need to move the sixth marker on chr 2 to chr 3, and
the fifth marker on chr 3 to chr 2.  We need to figure
out which markers these are.

\usercolor
\verb|pull.map(badorder, chr=2)| \\
\verb|pull.map(badorder, chr=3)| 
\normalcolor

Now we can use the function \verb-movemarker- to move the markers.
It seems like they should be exactly switched.

\usercolor \verb|badorder <- movemarker(badorder, "D2M937", 3, 48)| \\
\verb|badorder <- movemarker(badorder, "D3M160", 2, 28.8)| \normalcolor

Now look at the recombination fractions again.

\usercolor \verb|plot.rf(badorder, chr=2:3)| \normalcolor

\item We can check the marker order on chr 1.  The function
\verb-ripple- will consider all permutations of a sliding window of
adjacent markers.  A quick-and-dirty approach is to count the number
of obligate crossovers for each possible order, to find the order with
the minimum number of crossovers.  A more refined, but also more
computationally intensive, approach is to 
re-estimate the genetic map for each order, calculating LOD
scores (log$_{10}$ likelihood ratios) relative to the initial order.
(This may be done with allowance for the presence of genotyping
errors.)  The default approach is the quick-and-dirty method.

The following checks the marker order on chr 1, permuting
groups of six contiguous markers.  

\usercolor
\verb|rip1 <- ripple(badorder, chr=1, window=6)| \\
\verb|summary(rip1)| 
\normalcolor

In the summary output, markers 9--11 clearly need to be flipped.
There also seems to be a problem with the order of markers 4--6.  

\item The following performs the likelihood analysis, permuting groups of
three adjacent markers, assuming a genotyping error rate of
1\%.  It's considerably slower, but more trustworthy.
 
\usercolor 
\verb|rip2 <- ripple(badorder, chr=1, window=3, err=0.01, method="likelihood")| \\
\verb|summary(rip2)|
\normalcolor

Note that positive LOD scores indicate that the alternate order has a
higher likelihood than the original.

\item We can switch the order of markers 9--11 with the function
\verb-switch.order- (which works only for a single chromosome) and
then re-assess the order.  Note that the second row of \verb-rip1-
corresponds to the improved order.  

\usercolor
\verb|badorder.rev <- switch.order(badorder, 1, rip1[2,])| \\
\verb|rip1r <- ripple(badorder.rev, chr=1, window=6)| \\
\verb|summary(rip1r)|
\normalcolor

It looks like the marker pairs (5,6) and (1,2) should each be
inverted.  We use \verb-switch.order- again, and then check marker
order using the likelihood method.  

\usercolor
\verb|badorder.rev <- switch.order(badorder.rev, 1, rip1r[2,])|  \\
\verb|rip2r <- ripple(badorder.rev, chr=1, window=3, err=0.01)| \\
\verb|summary(rip2r)| 
\normalcolor

It's probably best to start out using the quick-and-dirty method, with
a large window size, to find the marker order with the minimum number
of obligate crossovers, and then refine that order using the slower,
but more trustworthy, likelihood method.

\item We can look again at the recombination fractions for this
chromosome.

\usercolor
\verb|badorder.rev <- est.rf(badorder.rev)| \\
\verb|plot.rf(badorder.rev, 1)| 
\normalcolor

\end{enumerate}




%\newpage
\vspace{12pt}
\noindent \textbf{Example 3: Listeria susceptibility} \vspace{6pt}
\nopagebreak

\noindent In order to demonstrate further uses of the function
\verb-scanone-, we consider some data on susceptibility to
\emph{Listeria monocytogenes\/} in mice (Boyartchuk et al., Nature
Genetics 27:259-260, 2001).  These data were kindly provided by Victor
Boyartchuk and Bill Dietrich.

\begin{enumerate}
\item Get access to the data and view some summaries.

\usercolor 
\verb|data(listeria)| \\
\verb|summary(listeria)| \\
\verb|plot(listeria)| \\
\verb|plot.missing(listeria)| 
\normalcolor

Note that in the missing data plot, gray pixels are partially missing
genotypes (e.g., a genotype may be known to be either AA or AB, but
not which).

The phenotype here is the survival time of a mouse (in hours)
following infection with \emph{Listeria monocytogenes}.  Individuals
with a survival time of 264 hours are those that recovered from the
infection.

\item We'll use the log survival time, rather than survival time, so
  we first need to create a new phenotype, which will end up as the
  third phenotype (after \verb-sex-).

\usercolor
\verb|listeria$pheno$logSurv <- log(listeria$pheno[,1])| \\ %$
\verb|plot(listeria)| \normalcolor

\item Estimate pairwise recombination fractions.

\usercolor \verb|listeria <- est.rf(listeria)| \\
\verb|plot.rf(listeria)| \\
\verb|plot.rf(listeria, chr=c(5,13))| 
\normalcolor

\item Re-estimate the genetic map. 
 
\usercolor 
\verb|newmap <- est.map(listeria, error.prob=0.01)| \\
\verb|plot.map(listeria, newmap)| \\
\verb|listeria <- replace.map(listeria, newmap)|
\normalcolor

\item Investigate genotyping errors; nothing gets flagged with a
  cutoff of 4, but one genotype is indicated with error LOD
  $\sim$3.8. 

\usercolor 
\verb|listeria <- calc.errorlod(listeria, error.prob=0.01)| \\
\verb|top.errorlod(listeria)| \\
\verb|top.errorlod(listeria, cutoff=3.5)| \\
\verb|plot.geno(listeria, chr=13, ind=61:70, cutoff=3.5)|
\normalcolor

Note that in the plot given by \verb-plot.geno-, for an intercross,
white = AA, gray = AB, black = BB, green = AA or AB, and orange = AB
or BB.

\item Now on to the QTL mapping.  Recall that the phenotype
distribution shows a clear departure from the standard assumptions for
interval mapping; 30\% of the mice survived longer than 264 hours, and
were considered recovered from the infection.

One approach for these data is to use the two-part model considered by
Boyartchuk et al.\ (2001).  In this model, a mouse with genotype $g$
has probability $p_g$ of surviving the infection.  If it does die, its
log survival time is assumed to be distributed
normal($\mu_g$,$\sigma^2$).  Analysis proceeds by maximum likelihood
via an EM algorithm.  Three LOD scores are calculated.  LOD($p,\mu$)
is for the test of the null hypothesis $p_g \equiv p$ and $\mu_g
\equiv \mu$.  LOD($p$) is for the test of the hypothesis $p_g \equiv
p$ but the $\mu$ are allowed to vary.  LOD($\mu$) is for the test of
the hypothesis $\mu_g \equiv \mu$ but the $p$ are allowed to vary.

The function \verb-scanone- will fit the above model when the argument
\verb-model="2part"-.  One must also specify the argument
\verb-upper-, which indicates whether the spike in the phenotype is
the maximum phenotype (as it is with this phenotype; take
\verb-upper=TRUE-) or the minimum phenotype (take \verb-upper=FALSE-).
For this model, only the EM algorithm has been implemented so far.

\usercolor
\verb|listeria <- calc.genoprob(listeria, step=2)| \\
\verb|out.2p <- scanone(listeria, pheno.col=3, model="2part", upper=TRUE)|
\normalcolor

Note the use of the argument \verb-pheno.col- to indicate the
phenotype column to use for the analysis.  We can also refer to the
phenotype column by name: \verb-pheno.col="logSurv"-.

Because the two-part model has three extra parameters, the
appropriate LOD threshold is higher---around 4.5 rather than 3.5.  The
three different LOD curves are in columns 3--5 of the output.  

\usercolor
\verb|summary(out.2p)| \\
\verb|summary(out.2p, threshold=4.5)| 
\normalcolor

Alternatively, we may use \verb-format="allpeaks"-, in which case it
displays the maximum LOD score or each column, with the position at
which each was maximized.  You may provide either one threshold, which
would be applied to all LOD score columns, or a separate threshold for
each column.

\usercolor
\verb|summary(out.2p, format="allpeaks", threshold=3)| \\
\verb|summary(out.2p, format="allpeaks", threshold=c(4.5,3,3))| 
\normalcolor

\item By default, \verb-plot.scanone- will plot the first LOD score
  column.  Alternatively, we may indicate another column to plot with
  the \verb-lodcolumn- argument.  Or we can plot up to three LOD
  scores at once by giving a vector.

\usercolor
\verb|plot(out.2p)| \\
\verb|plot(out.2p, lodcolumn=2)| \\
\verb|plot(out.2p, lodcolumn=1:3, chr=c(1,5,13,15))|
\normalcolor

Note that the locus on chr 1 shows effect mostly on the
mean time-to-death, conditional on death; the locus on chr 5
shows effect mostly on the probability of survival; and the loci on
chr 13 and 15 shows some effect on each. 

\item Permutation tests may be performed as before.  The output will
have three columns, corresponding to the three LOD scores.

\usercolor 
\verb|operm.2p <- scanone(listeria, model="2part", pheno.col=3,| \\
\verb|                    upper=TRUE, n.perm=25)| \\
\verb|summary(operm.2p, alpha=0.05)|
\normalcolor 

We may again use the permutation results in \verb-summary.scanone- to
have thresholds calculated automatically and to obtain
genome-scan-adjusted p-values, but of course we would want to have
performed more than 25 permutations.

\usercolor
\verb|summary(out.2p, format="allpeaks", perms=operm.2p,| \\
\verb|        alpha=0.05, pvalues=TRUE)|
\normalcolor

\item Alternatively, one may perform separate analyses of the log
survival time, conditional on death, and the binary phenotype
survival/death.  First we set up these phenotypes.

\usercolor
\verb|y <- listeria$pheno$logSurv| \\ 
\verb|my <- max(y, na.rm=TRUE)| \\
\verb|z <- as.numeric(y==my)| \\
\verb|y[y==my] <- NA| \\
\verb|listeria$pheno$logSurv2 <- y| \\
\verb|listeria$pheno$binary <- z| \\
\verb|plot(listeria)|
\normalcolor

We use standard interval mapping for the log survival time conditional
on death; the results are slightly different from LOD($\mu$).

\usercolor
\verb|out.mu <- scanone(listeria, pheno.col=4)| \\
\verb|plot(out.mu, out.2p, lodcolumn=c(1,3), chr=c(1,5,13,15), col=c("blue","red"))|
\normalcolor

We can use \verb-scanone- with \verb-model="binary"- to analyze the
binary phenotype.  Again, the results are only slight different from
LOD($p$). 

\usercolor
\verb|out.p <- scanone(listeria, pheno.col=5, model="binary")| \\
\verb|plot(out.p, out.2p, lodcolumn=c(1,2), chr=c(1,5,13,15), col=c("blue","red"))|
\normalcolor

The argument \verb-pheno.col- in \verb-scanone- can actually take a
vector of numeric phenotype values, and not just an indicator to a
phenotype column, and so we could have performed the binary trait
analysis without first pasting the binary phenotype into the
\verb-listeria- object, as follows.

\usercolor
\verb|out.p.alt <- scanone(listeria, pheno.col=as.numeric(listeria$pheno$T264==264),|\\
\verb|                     model="binary")|
\normalcolor

\item A further approach is to use a non-parametric form of interval
mapping.  R/qtl uses an extension of the Kruskal-Wallis test
statistic.  Use \verb-scanone- with \verb-model="np"-.  In this case,
the argument \verb-method- is ignored; the analysis method is much
like Haley-Knott regression.  If the argument \verb-ties.random=TRUE-,
tied phenotypes are ranked at random.  If \verb-ties.random=FALSE-,
tied phenotypes are given the average rank and a correction is applied
to the LOD score.

\usercolor
\verb|out.np1 <- scanone(listeria, model="np", ties.random=TRUE)| \\
\verb|out.np2 <- scanone(listeria, model="np", ties.random=FALSE)| 

\verb|plot(out.np1, out.np2, col=c("blue","red"))| \\
\verb|plot(out.2p, out.np1, out.np2, chr=c(1,5,13,15))|
\normalcolor

Note that the significance threshold for the non-parametric genome
scan will be quite a bit smaller than that for the two-part model.
The two approaches for dealing with ties give basically the same
results.  Randomizing ties for the non-parametric approach can give
quite variable results in the case of a great number of ties, and so
we would recommend the use of \verb-ties.random=FALSE- in this case.

\end{enumerate}



\vspace{12pt}
%\newpage

\noindent \textbf{Example 4: Covariates in QTL mapping} \vspace{6pt}
\nopagebreak

\noindent As a further example, we illustrate the use of covariates in
QTL mapping.  We consider some simulated backcross data.

\begin{enumerate}

\item Get access to the data.

\usercolor
\verb|data(fake.bc)| \\
\verb|summary(fake.bc)| \\
\verb|plot(fake.bc)| 
\normalcolor

\item Perform genome scans for the two phenotypes without covariates.
Here we consider two phenotypes, scanned individually.

\usercolor
\verb|fake.bc <- calc.genoprob(fake.bc, step=2.5)| \\
\verb|out.nocovar <- scanone(fake.bc, pheno.col=1:2)|
\normalcolor

\item Perform genome scans with sex as an additive covariate.
Note that the covariates must be numeric.  Factors may have to be
converted.

\usercolor
\verb|sex <- fake.bc$pheno$sex| \\
\verb|out.acovar <- scanone(fake.bc, pheno.col=1:2, addcovar=sex)|
\normalcolor

Here, the average phenotype is allowed to be different in the two
sexes, but the effect of the putative QTL is assumed to be the same in
the two sexes.

\item Note that the use of sex as an additive covariate resulted in an
  increase in the LOD scores for phenotype 1, but resulted  in a
  decreased LOD score at the chr 5 locus for phenotype 2.

\usercolor
\verb|summary(out.nocovar, threshold=3, format="allpeaks")| \\
\verb|summary(out.acovar, threshold=3, format="allpeaks")| 

\verb|plot(out.nocovar, out.acovar, chr=c(2, 5))| \\
\verb|plot(out.nocovar, out.acovar, chr=c(2, 5), lodcolumn=2)| 
\normalcolor

\item Let us now perform genome scans with sex as an interactive
  covariate, so that the QTL is allowed to be different in the two sexes.

\usercolor
\verb|out.icovar <- scanone(fake.bc, pheno.col=1:2, addcovar=sex, intcovar=sex)|
\normalcolor

\item The LOD score in the output is for the comparison of the full
  model with terms for sex, QTL and QTL$\times$sex interaction to the reduced
  model with just the sex term.  Thus, the degrees of freedom
  associated with the LOD score is 2 rather than 1, and so larger LOD
  scores will generally be obtained.

\usercolor
\verb|summary(out.icovar, threshold=3, format="allpeaks")|
\normalcolor

\usercolor
\verb|plot(out.acovar, out.icovar, chr=c(2,5), col=c("blue", "red"))| \\
\verb|plot(out.acovar, out.icovar, chr=c(2,5), lodcolumn=2,| \\
\verb|     col=c("blue", "red"))|
\normalcolor

\item The difference between the LOD score with sex as an interactive
  covariate and the LOD score with sex as an additive covariate
  concerns the test of the QTL$\times$sex interaction: does the QTL
  have the same effect in both sexes?  The differences, and a plot of
  the differences, may be obtained as follows.

\usercolor
\verb|out.sexint <- out.icovar - out.acovar| \\
\verb|plot(out.sexint, lodcolumn=1:2, chr=c(2,5), col=c("green", "purple"))|
\normalcolor

The green and purple curves are for the first and second phenotypes,
respectively.  

\item To test for the QTL$\times$sex interaction, we may perform a
  permutation test.  This is not perfect, as the permutation test
  eliminates the effect of the QTL, and so we must assume that the
  distribution of the LOD score for the QTL$\times$sex interaction is
  the same in the presence of a QTL as under the global null
  hypothesis of no QTL effect.

  The permutation test requires some care.  We must perform
  separate permutations with sex as an additive covariate and with sex
  as an interactive covariate, but we must ensure, by setting the
  ``seed'' for the random number generator, that they use matched
  permutations of the data.

  For the sake of speed, we will use Haley-Knott regression, even
  though the results above were obtained by standard interval
  mapping. Also, we will perform just 100 permutations, though 1000
  would be preferred.

\usercolor
\verb|seed <- ceiling(runif(1, 0, 10^8))| \\
\verb|set.seed(seed)| \\
\verb|operm.acovar <- scanone(fake.bc, pheno.col=1:2, addcovar=sex,| \\
\verb|                        method="hk", n.perm=100)| \\
\verb|set.seed(seed)| \\
\verb|operm.icovar <- scanone(fake.bc, pheno.col=1:2, addcovar=sex,| \\
\verb|                        intcovar=sex, method="hk", n.perm=100)|
\normalcolor

Again, the differences concern the QTL$\times$sex interaction.

\usercolor
\verb|operm.sexint <- operm.icovar - operm.acovar|
\normalcolor

We can use \verb-summary- to get the genome-wide LOD thresholds.

\usercolor
\verb|summary(operm.sexint, alpha=c(0.05, 0.20))|
\normalcolor

We can also use these results to look at evidence for QTL$\times$sex
interaction in our initial scans.

\usercolor
\verb|summary(out.sexint, perms=operm.sexint, alpha=0.1,| \\
\verb|        format="allpeaks", pvalues=TRUE)|
\normalcolor


\end{enumerate}


%\vspace{12pt}
\newpage
\noindent \textbf{Example 5: Multiple QTL mapping} \vspace{6pt}
\nopagebreak

\label{example5}

We return to the \verb-hyper- data to illustrate some of the more
advanced methods for exploring multiple QTL models.  Note that the
multiple QTL mapping features are currently implemented only for
multiple imputation and Haley-Knott regression.  We use multiple
imputation here, as Haley-Knott regression performs poorly in the case
of selective genotyping, which was used for the \verb-hyper- data.


\begin{enumerate}
\item First, let us delete everything in our workspace and then
  re-load the \verb-hyper- data.

\usercolor \verb|rm(list=ls())| \\
\verb|data(hyper)| \normalcolor

\item We will be using the multiple imputation method throughout this
  example, and so we first need to perform the imputations.  Recall
  that more imputations give more precise results, but take more
  time and memory.  To speed things along, we will use only 16
  imputations, even though much more would be needed for a
  definitive analysis.  The small number of imputations will make the
  following results somewhat unpredictable.

\usercolor \verb|hyper <- sim.geno(hyper, step=2.5, n.draws=16, err=0.01)|
\normalcolor

\item We first perform a single-QTL genome scan and inspect the
  results.

\usercolor \verb|out1 <- scanone(hyper, method="imp")| \\
\verb|plot(out1)|
\normalcolor

As you may recall from the results in Example 1, we have clear evidence 
for a QTL on chr 4, and strong evidence for a QTL on chr 1.
The LOD curve on chr 1 has an interesting double peak, suggestive of
possibly two QTL.

There is a hint of further loci on chr 6 and 15 and elsewhere.

\item In the presence of a large-effect QTL, as seen on chr 4,
  one may wish to repeat the scan, controlling for that locus.  This
  can make the loci with more modest effect more apparent.  

  A simple (but rough) approach is to pull out the genotypes for a
  marker near the peak locus, and use that marker as an additive
  covariate in a single-QTL scan.  The peak marker for these data was
  D4Mit164:

  \usercolor \verb|max(out1)| \normalcolor

  If the peak LOD score is not at a marker, we may use
  \verb-find.marker- to identify the marker closest to the LOD peak.  

  \usercolor \verb|find.marker(hyper, 4, 29.5)| \normalcolor

\item  The function \verb-pull.geno- may be used to pull out the genotype
  data for that marker, but we'll see that most individuals were not
  typed at D4Mit164.

  \usercolor \verb|g <- pull.geno(hyper)[,"D4Mit164"]| \\
  \verb|mean(is.na(g))| \normalcolor

  We may fill in the genotype data using a single imputation,
  and then use those imputed genotypes as if they were observed.  This
  is not ideal; we'll do this analysis properly below.

  \usercolor \verb|g <- pull.geno(fill.geno(hyper))[,"D4Mit164"]|
  \normalcolor 

\item Now we perform the genome scan, controlling for the chr 4
  locus. (Note that in an intercross, we would have to re-code the genotype
  data to be a two-column numeric matrix.)

  \usercolor
  \verb|out1.c4 <- scanone(hyper, method="imp", addcovar=g)|
  \normalcolor

  We can plot the results together with the original genome scan.

\usercolor \verb|plot(out1, out1.c4, col=c("blue", "red"))| \normalcolor

  The LOD curve on chr 1 went up quite a bit.  (And, of course,
  the LOD curve on chr 4 went down to near 0.)  To see the
  effect of controlling for the chr 4 locus more clearly, we
  can plot the differences between the LOD scores.

  \usercolor \verb|plot(out1.c4 - out1, ylim=c(-3,3))| \\
  \verb|abline(h=0, lty=2, col="gray")| \normalcolor

\item We may also look for loci that interact with the chr 4
  locus, by including marker D4Mit164 as
  an interactive covariate. 

  \usercolor
  \verb|out1.c4i <- scanone(hyper, method="imp", addcovar=g, intcovar=g)| 
\normalcolor

  The difference between these LOD scores and those obtained with D4Mit164 as a
  strictly additive covariate indicates evidence for an interaction with the
  chr 4 locus.

  \usercolor \verb|plot(out1.c4i - out1.c4)| \normalcolor

  There is nothing particularly interesting here.

\item Now let us perform a 2d scan.  This will take a few minutes, as
  we're doing the scan at a 2.5~cM step size.

\usercolor \verb|out2 <- scantwo(hyper, method="imp")| \normalcolor

\item Let us look at some summaries for the \verb-scantwo- results.
  Recall that we need to provide five thresholds (see Example~1,
  item~\ref{scantwo} on page~\pageref{scantwo}).
  We'll ignore the threshold on the epistasis LOD score, $T_i$, and
  use the thresholds suggested above.

\usercolor \verb|summary(out2, thr=c(6.0, 4.7, Inf, 4.7, 2.6))|
\normalcolor

Your results may be different from mine, since we are using so few
imputations, but I see evidence for loci on chr 1 and 4 (which
don't appear to interact) and loci on chr 6 and 15 (which do
show evidence of epistasis).  

This didn't pick up evidence for two QTL on chr 1; we can look
directly at the chr 1 results as follows.

\usercolor \verb|summary( subset(out2, chr=1) )| \normalcolor

The LOD score for a second, additive QTL on chr 2
($\lod_{av1}$) is $\sim$1.6; not strong, but not uninteresting. 

Evidence for an interaction between loci on chr 7 and 15 had
been previously reported.  Those results may be inspected as follows.

\usercolor \verb|summary( subset(out2, chr=c(7,15)) )| \normalcolor

Again, this is interesting but not strong.

\item Let us look at some plots of the \verb-scantwo- results.  First
  we make the standard plot with selected chromosomes; the upper
  triangle contains $\lod_i$ and the lower triangle contains $\lod_f$.

\usercolor \verb|plot(out2, chr=c(1,4,6,7,15))| \normalcolor

The arguments \verb-lower- and \verb-upper- may be used to change what
is plotted in the upper and lower triangles.  For example, with
\verb:lower="cond-int": , $\lod_{fv1}$ (evidence for a second QTL,
allowing for epistasis) is displayed in the lower triangle, while with
\verb:lower="cond-add":, $\lod_{av1}$ (evidence for a second QTL,
assuming no epistasis) is displayed.

\usercolor \verb|plot(out2, chr=1, lower="cond-add")| \\
\verb|plot(out2, chr=c(6,15), lower="cond-int")| \\
\verb|plot(out2, chr=c(7,15), lower="cond-int")| \normalcolor

Again, evidence for a second QTL on chr 1 is not strong.
Evidence for interacting QTL on chr 6 and 15 is quite strong;
the 7$\times$15 interaction is not.  

\item We can also perform the 2d scan conditional on the chr 4
  locus.  We'll do this just for chr 1, 6, 7, and 15, to save
  time.

\usercolor \verb|out2.c4 <- scantwo(hyper, method="imp", addcovar=g, chr=c(1,6,7,15))| \normalcolor

  If we look at the same summaries as before, we see decreased
  evidence for a second QTL on chr 1 and for the 7$\times$15
  interaction, but increased evidence for the 6$\times$15 interaction.

\usercolor \verb|summary(out2.c4, thr=c(6.0, 4.7, Inf, 4.7, 2.6))| \\
\verb|summary( subset(out2.c4, chr=1) )| \\
\verb|summary( subset(out2.c4, chr=c(7,15)) )| 
\normalcolor

  The sort of plots we made before remain interesting.

\usercolor \verb|plot(out2.c4)| \\
\verb|plot(out2.c4, chr=1, lower="cond-int")| \\
\verb|plot(out2.c4, chr=c(6,15), lower="cond-int")| \\
\verb|plot(out2.c4, chr=c(7,15), lower="cond-int")| \normalcolor

  We can also look at the differences in the LOD scores, to see how
  much conditioning on D4Mit164 has affected the results.  We need to
  subset our original results, since we only scanned selected
  chromosomes in the conditional analysis.  The \verb-allow.neg-
  argument is used to allow negative LOD scores in the \verb-scantwo-
  plot, as they would generally be replaced with 0.  

\usercolor \verb|out2sub <- subset(out2, chr=c(1,6,7,15))| \\
\verb|plot(out2.c4 - out2sub, allow.neg=TRUE, lower="cond-int")|
\normalcolor

\item Now let us turn to the fit of multiple-QTL models.  The function
  \verb-fitqtl- is used to fit a specific model. 

  One must first pull out the data on fixed QTL locations using
  \verb-makeqtl-.  We will consider the possibility of two QTL on
  chr 1, but will ignore the putative QTL on chr 7.

\usercolor \verb|qc <- c(1, 1, 4, 6, 15)| \\
\verb|qp <- c(43.3, 78.3, 30.0, 62.5, 18.0)| \\
\verb|qtl <- makeqtl(hyper, chr=qc, pos=qp)| \normalcolor 

  We also create a ``formula'' which indicates which QTL are to be
  included in the fit and which interact; the colon (:) indicates an
  interaction.  

\usercolor \verb|myformula <- y ~ Q1+Q2+Q3+Q4+Q5 + Q4:Q5| \normalcolor  

  We can now fit a model, including the 6$\times$15 interaction, and
  get a summary of the results.  

\usercolor
\verb|out.fq <- fitqtl(hyper, qtl=qtl, formula = myformula)| \\
\verb|summary(out.fq)|
\normalcolor

The first part of the summary describes the overall fit; the LOD score
of $\sim$23 is the log$_{10}$ likelihood ratio comparing the full
model to the null model.

The second part of the summary gives results dropping one term at a
time from the model.  In the presence of an interaction, if a term
included in the interaction is omitted, the interaction is also
omitted, and so the rows for the loci on chr 6 and 15
indicate 2 degrees of freedom.

\item One may also use \verb-fitqtl- to get estimated effects of the
  QTL in the context of the multiple-QTL model.  We can use
  \verb-drop=FALSE-, so that the ``drop one at a time'' part of the
  analysis is not performed, and \verb-get.ests=TRUE- to get the
  estimated effects.

\usercolor
\verb|out.fq <- fitqtl(hyper, qtl=qtl, formula = myformula, drop=FALSE, get.ests=TRUE)| \\
\verb|summary(out.fq)|
\normalcolor

The estimated effects are the differences between the heterozygote and
homozygote groups.  The interaction effect is the difference
between the differences.

\item The function \verb-refineqtl- can be used to refine the
  estimated positions of the QTL in the context of the multiple-QTL
  model.  A QTL object may be provided, or one may specify the
  chromosomes and positions, as in \verb-makeqtl-; we'll use the
  former approach.

\usercolor
\verb|revqtl <- refineqtl(hyper, qtl=qtl, formula = myformula)|
\normalcolor

The output is a QTL object, like \verb-qtl-; typing its name gives a
brief summary.  

\usercolor \verb|revqtl| \normalcolor

A couple of the QTL moved, but none by very much.

One may use the \verb-plot.qtl- function to plot the
locations of the QTL on the genetic map.

\usercolor \verb|plot(revqtl)| \normalcolor

We can re-run \verb-fitqtl- to get a fit with the new positions; the
overall LOD score should have increased slightly.  (For me, it
increased from 23.0 to 23.7.)

\usercolor
\verb|out.fq2 <- fitqtl(hyper, qtl=revqtl, formula=myformula)| \\
\verb|summary(out.fq2)|
\normalcolor

\item The \verb-scanqtl- function is used to perform general genome
  scans in the context of a multiple QTL model.  It is quite flexible,
  but not simple to use.  For most purposes, one may focus on the
  functions \verb-addqtl- and \verb-addpair-, which scan for an
  additional QTL or pair of QTL, respectively, to add to a
  multiple-QTL model.

  We will first use \verb-addqtl- to perform a more precise version of
  our genome scan conditional on the chr 4 locus.  Previously, we had
  conditioned on imputed genotypes at a marker near the LOD peak on
  chr 4.  With \verb-addqtl- we can do this properly: take proper
  account of the missing genotype information at the chr 4 locus,
  rather than taking genotypes from a single imputation as if they had
  been observed.

  The \verb-addqtl- function is much like \verb-fitqtl-, taking a QTL
  object and formula as arguments.  If the formula is omitted,
  all loci are assumed to be additive.  The additional QTL to be
  scanned may be included in the formula; if there are 5 QTL in the
  input QTL object, refer to the new QTL as \verb-Q6-.  This allows a
  scan with the new QTL interacting with one or more of the current
  QTL.  If the new QTL is not included in the formula, it is assumed
  to be strictly additive.

  The following performs a scan on all chromosomes, controlling solely
  for the QTL on chromosome 4.  (This is the third QTL in the QTL
  object \verb-revqtl-, and so we may use as the formula either
  \verb-y~Q3- or \verb-y~Q3+Q6-.  The former is allowed, as an
  additional additive QTL is assumed.)

\usercolor
\verb|out1.c4r <- addqtl(hyper, qtl=revqtl, formula=y~Q3)|
\normalcolor

The output is of the same form as produced by the \verb-scanone-
function, and so we may use the same plot and summary functions as are
used for \verb-scanone- results.  (Note that the LOD scores produced
by \verb-addqtl- are relative to the model specified in the formula,
omitting any terms including the additional QTL being scanned, rather
than relative to the null model.).

We may now plot these results with those obtained earlier.  The
results are actually not too different. 

\usercolor
\verb|plot(out1.c4, out1.c4r, col=c("blue", "red"))| 
\normalcolor

It may be more informative to plot the differences

\usercolor
\verb|plot(out1.c4r - out1.c4, ylim=c(-1.7, 1.7))| \\
\verb|abline(h=0, lty=2, col="gray")|
\normalcolor

\item The function \verb-addpair- may be used to perform a 2d scan for
  an additional pair of QTL, conditioning on the locus on chr 4.  If
  the new QTL are not specified in the formula, a scan as in
  \verb-scantwo- is performed (that is, for each possible pair of
  positions for the new QTL, we fit a model in which the two new QTL
  interact and one in which they are additive).

\usercolor
\verb|out2.c4r <- addpair(hyper, qtl=revqtl, formula=y~Q3, chr=c(1,6,7,15))|
\normalcolor

The results are of the same form as produced by \verb-scantwo-, and

We can plot the difference between these results and our previous
results.

\usercolor
\verb|plot(out2.c4r - out2.c4, lower="cond-int", allow.neg=TRUE)|
\normalcolor

Again, things have not changed dramatically.

\item The most interesting use of \verb-addqtl- and \verb-addpair- is
  to scan for additional loci, starting with our five-QTL model (with
  the loci on 6 and 15 interacting).

  First, we scan for an additional additive QTL.

\usercolor
\verb|out.1more <- addqtl(hyper, qtl=revqtl, formula=myformula)| \\
\verb|plot(out.1more)| 
\normalcolor

There is not much evidence for an additional QTL.

\item We may next scan for an additional QTL that interacts with one
  of the QTL in our model, such as the QTL on chr 15.  This may be done
  by indicating the interaction in the formula, using \verb-Q6- to
  specify the new QTL, since there are five QTL in the \verb-revqtl- object.

\usercolor
\verb|out.iw4 <- addqtl(hyper, qtl=revqtl, formula=y~Q1+Q2+Q3+Q4+Q5+Q4:Q5+Q6+Q5:Q6)| \\
\verb|plot(out.iw4)| 
\normalcolor

The LOD scores are just slightly higher, but there are two degrees of
freedom in the test.  There's nothing particularly exciting here.

\item Now, let us scan for an additional pair.  This will take
  quite a bit of time, so let's focus on a few chromosomes: 2, 5, 7
  and 15.

\usercolor
\verb|out.2more <- addpair(hyper, qtl=revqtl, formula=myformula, chr=c(2,5,7,15))| 
\normalcolor

Again, the results are of the form produced by \verb-scantwo-, and so
we may use the same plot and summary functions.

\usercolor
\verb|plot(out.2more, lower="cond-int")|
\normalcolor

Again, there's nothing particularly exciting.  

\item Another function of interest is \verb-addint-, for testing the
  addition of each possible pairwise interactions, one at a time, to a
  multiple-QTL model.  

\usercolor
\verb|out.ai <- addint(hyper, qtl=revqtl, formula=myformula)| \\
\verb|out.ai|
\normalcolor

The results contain one row per interaction, and contain the same sort
of information as produced by in the drop-one analysis of
\verb-fitqtl-.  As the base model (in \verb-myformula-) contains an
interaction between the loci on chr 6 and 15, that particular
interaction is not tested.

\item We should mention the functions for manipulating
  QTL objects (produced by \verb-makeqtl-): \verb-addtoqtl-,
  \verb-dropfromqtl-, \verb-replaceqtl-, and \verb-reorderqtl-.  

  If the use of \verb-addqtl- and \verb-addpair- had indicated
  evidence for additional QTL, one could add them to the QTL object
  with \verb-addtoqtl-.  As input, one provides the cross, the QTL
  object, and the chromosomes and positions of the QTL to be added.

\usercolor
\verb|qtl2 <- addtoqtl(hyper, revqtl, 7, 53.6)| \\
\verb|qtl2|
\normalcolor

  A QTL may be removed with \verb-dropfromqtl-.  One provides either
  the numeric index within the object, the QTL name, or the chromosome and
  position of the QTL to be dropped.

\usercolor
\verb|qtl3 <- dropfromqtl(qtl2, index=2)| \\
\verb|qtl3|
\normalcolor

  We can use \verb-replaceqtl- to move a particular QTL to a new
  position.  One must provide the index of the QTL to be replaced.

\usercolor
\verb|qtl4 <- replaceqtl(hyper, qtl3, indextodrop=1, chr=1, pos=50)| \\
\verb|qtl4|
\normalcolor

  We use \verb-reorderqtl- to change the order of the loci within a
  QTL object.

\usercolor
\verb|qtl5 <- reorderqtl(qtl4, c(1:3,5,4))| \\
\verb|qtl5|
\normalcolor

\item Finally, we consider an automated model selection procedure with
  a stepwise search algorithm, using the function \verb-stepwiseqtl-.
  The function seeks to optimize a penalized LOD score criterion,
  which is the LOD score for a model (relative to the null model with
  no QTL) with penalties on each QTL main effect and a separate
  penalty on interactions.  

  Actually, we include include two penalties on interactions, a light
  penalty and a heavy penalty.  We focus on models with possible
  pairwise interactions among QTL, and with a hierarchical
  structure in which the inclusion of an interaction term requires
  the inclusion of both of the corresponding main effects terms.  Such
  a model may be represented by a graph in which vertices (dots)
  represent QTL and edges (line segments between the dots) represent
  interactions between QTL.  In the penalized LOD score considered by
  \verb-stepwiseqtl-, each disconnected component of a model is
  allowed one light interaction penalty; all other interactions are
  assigned the heavy penalty.  

  The three penalties may be calculed from permutation results with
  \verb-scantwo-, using the function \verb-calc.penalties-.  We will
  use default penalties derived by computer simulation: (2.69, 2.62,
  1.19) for a mouse backcross, or (3.52, 4.28, 2.69) for a mouse
  intercross.  (The penalties are in the order (main, heavy
  interaction, light interaction).)

  First, let us apply \verb-stepwiseqtl-, considering only additive
  QTL models (with \verb-additive.only=TRUE-.  The algorithm performs
  forward selection up to a model with a given number of QTL
  (specified by the argument \verb-max.qtl-; we'll use 6), followed by
  backward elimination.

\usercolor
\verb|stepout.a <- stepwiseqtl(hyper, additive.only=TRUE, max.qtl=6)| \\
\verb|stepout.a|
\normalcolor

   I obtained a model with two QTL, with one QTL on each of chr 1 and
   4.

   Now let's re-run the analysis, allowing for the possibility of
   interactions among the QTL.

\usercolor
\verb|stepout.i <- stepwiseqtl(hyper, max.qtl=6)| \\
\verb|stepout.i|
\normalcolor

  I obtained a model with four QTL, including one on each of chr 1,
  4, 6 and 15, and including an interaction between the loci on chr 6
  and 15. 

\item Note that all of the above could be performed using
  Haley-Knott regression rather than multiple imputation.  Just three
  changes need to be made.  

  First, one needs to run
  \verb-calc.genoprob- rather than \verb-sim.geno-, to calculate the
  QTL genotype probabilities rather than perform imputations.

  Second, in a call to \verb-makeqtl-, use the argument
  \verb-what="prob"-, so that the genotype probabilities are placed in
  the object rather than imputations.

  Third, in calls to \verb-fitqtl-, \verb-addqtl-, \verb-addpair-,
  etc., use \verb-method="hk"-.

\end{enumerate}




%\vspace{12pt}
\newpage
\noindent \textbf{Example 6: Internal data structure} \vspace{6pt}
\nopagebreak

\label{example6}

\noindent Finally, let us briefly describe the rather complicated data
structure that R/qtl uses for QTL mapping experiments.  This will be
rather dull, and will require a good deal of familiarity with the R
(or S) language.  The choice of data structure required some balance
between ease of programming and simplicity for the user interface.
The syntax for references to certain pieces of the internal data can
become extremely complicated.

\begin{enumerate}

\item Get access to some sample data.

\usercolor \verb|data(fake.bc)| \normalcolor

\item First, the object has a ``class,'' which indicates that it
corresponds to data for an experimental cross, and gives the cross
type.  By having class \verb-cross-, the functions \verb-plot- and
\verb-summary- know to send the data to \verb-plot.cross- and
\verb-summary.cross-.

\usercolor \verb|class(fake.bc)| \normalcolor

\item Every \verb-cross- object has two components, one containing the
genotype data and genetic maps and the other containing the phenotype
data.

\usercolor \verb|names(fake.bc)| \normalcolor

\item The phenotype data is simply a matrix (more strictly a
data.frame) with rows corresponding to individuals and columns
corresponding to phenotypes.  

\usercolor \verb|fake.bc$pheno[1:10,]| \normalcolor %$

\item  The genotype data is a list with components corresponding to
chromosomes.  Each chromosome has a name and a class.  The class for a
chromosome is either \verb-"A"- or \verb-"X"-, according to whether it
is an autosome or the X chromosome.  

\usercolor 
\verb|names(fake.bc$geno)| \\ %$
\verb|sapply(fake.bc$geno, class)| %$
\normalcolor

\item Each component of \verb-geno- contains two components,
\verb-data- (containing the marker genotype data) and \verb-map-
(containing the positions of the markers, in cM).

\usercolor
\verb|names(fake.bc$geno[[3]])| \\ %$
\verb|fake.bc$geno[[3]]$data[1:5,]| \\ 
\verb|fake.bc$geno[[3]]$map|
\normalcolor

That's it for the raw data.

\item When one runs \verb-calc.genoprob-, \verb-sim.geno-,
\verb-argmax.geno- or \verb-calc.errorlod-, the output is the input
cross object with the derived data attached to each component (the
chromosomes) of the \verb-geno- component.  

\usercolor
\verb|names(fake.bc$geno[[3]])| \\ %$
\verb|fake.bc <- calc.genoprob(fake.bc, step=10, err=0.01)| \\
\verb|names(fake.bc$geno[[3]])| \\ %$ 
\verb|fake.bc <- sim.geno(fake.bc, step=10, n.draws=8, err=0.01)| \\
\verb|names(fake.bc$geno[[3]])| \\ %$ 
\verb|fake.bc <- argmax.geno(fake.bc, step=10, err=0.01)| \\
\verb|names(fake.bc$geno[[3]])| \\ %$
\verb|fake.bc <- calc.errorlod(fake.bc, err=0.01)| \\
\verb|names(fake.bc$geno[[3]])|%$
\normalcolor

\item Finally, when one runs \verb-est.rf-, a matrix containing the
pairwise recombination fractions and LOD scores is added to the cross
object.

\usercolor
\verb|names(fake.bc)| \\
\verb|fake.bc <- est.rf(fake.bc)| \\
\verb|names(fake.bc)|
\normalcolor

\end{enumerate}

\end{document}
